\documentclass[12pt]{article} % 

\usepackage{amssymb}
\usepackage{amsthm}
\usepackage{amsfonts}
\usepackage{amsmath}
\usepackage{enumerate}

\setlength{\oddsidemargin}{-0.15in}
\setlength{\topmargin}{-0.5in}
\setlength{\textwidth}{6.5in}
\setlength{\textheight}{9in}


\newcommand{\N}{\mathbb{N}}
\newcommand{\Z}{\mathbb{Z}}
\newcommand{\Q}{\mathbb{Q}}
\newcommand{\R}{\mathbb{R}}
\newcommand{\C}{\mathbb{C}}
\newcommand{\F}{\mathbb{F}}
\newcommand{\id}{{\rm id}}
\newcommand{\calE}{\mathcal{E}}
\newcommand{\s}{{\rm span}}
\newcommand{\rk}{{\rm rank}}
\newcommand{\hatF}{{\widehat{F}}}


\renewcommand\le{\leqslant}
\renewcommand\ge{\geqslant}


\begin{document} 
\noindent
\textbf{Math 309 -- Intermediate Linear Algebra \quad 
Homework 9 \hfill Your name}\\
\begin{center}
  (Due Friday, April 13)
\end{center}
\medskip

Each problem will be graded out of 10 points.

\vspace{1cm}

\begin{flushleft}

1.  Let $A = \left(\begin{matrix}
                    2 & -1.3 \\
                    2 & -1.6\\
                   \end{matrix}\right).$  Does $\lim_{m \to \infty}\limits A^m$ exist?  If so, compute it.\\
                   
\vspace{.5cm}                   

2.  Consider the system from Lecture 3/28 with states 1 and 2 that has transition matrix \[A = \left(\begin{matrix}
                                                                                                      0.9 & 0.4 \\
                                                                                                      0.1 & 0.6\\
                                                                                                     \end{matrix}\right)\]
 and initial probability vector \[w = \left(\begin{matrix}
                                            0.2 \\
                                            0.8\\
                                           \end{matrix}\right).\]
 In the long run, with what probability do we expect the system to be in state 1 and with what probability do we expect the system to be in state 2?\\                                                                                                                                               


\vspace{.8cm}

Two linear operators $S,T$ on a finite-dimensional vector space $V$ are said to be \textit{simultaneously diagonalizable} if there exists a basis $\beta$ of $V$ such that both $[S]_\beta$ and $[T]_\beta$ are diagonal matrices.\\                                    
\vspace{.3cm}


3.  If $S$ and $T$ are linear operators on a finite-dimensional vector space $V$ that commute and are both diagonalizable, then prove that $S$ and $T$ are simultaneously diagonalizable.\\

\vspace{.5cm}



4.  If $S$ and $T$ are linear operators on a finite-dimensional vector space $V$ that are simultaneously diagonalizable, then prove that $S$ and $T$ commute.


\vspace{.8cm}

5.  Prove that if a $1$-dimensional subspace $W$ of $\R^n$ contains a nonzero vector with all nonnegative entries, then $W$ contains a unique probability vector.


\vspace{.5cm}

6.  Let $A$ and $B$ be $n \times n$ transition matrices.  If $A$ is regular and $c$ is a real number such that $0 < c \le 1$, prove that $cA + (1-c)B$ is a regular transition matrix.\\


 



 
\end{flushleft}

\end{document}