\documentclass[12pt]{article} %

\usepackage{amssymb}
\usepackage{amsthm}
\usepackage{amsfonts}
\usepackage{amsmath}

\setlength{\oddsidemargin}{-0.15in}
\setlength{\topmargin}{-0.5in}
\setlength{\textwidth}{6.5in}
\setlength{\textheight}{9in}


\newcommand{\N}{\mathbb{N}}
\newcommand{\Z}{\mathbb{Z}}
\newcommand{\R}{\mathbb{R}}
\newcommand{\id}{{\rm id}}


\renewcommand\le{\leqslant}
\renewcommand\ge{\geqslant}


\begin{document}
\noindent
\textbf{Math 309 -- Intermediate Linear Algebra \quad
Homework 1 \hfill Xida Ren}\\
\begin{center}
  (Due Friday, February 2)
\end{center}
\medskip

\begin{flushleft}
\begin{itemize}
\item \textbf{Justify your work.  Do not skip steps!}
\item You may cite the result of an earlier problem on this homework.
\item Each problem will be graded out of 10 points.
\end{itemize}

\vspace{1cm}

1.  Prove that congruence modulo $n$ is an equivalence relation on the integers, where $n \in \N$.\\



\vspace{0.5cm}

We have proven in class that partitions induce equivalence relations. 

Congruence modulo $n$ is a partition of the integers into $n$ equivalence classes: we put an integer $i$ into the set numbered $k$ if $i \mod n \equiv k$.

Each integer belongs to at least one class because every integer has a remainder $\mod n$. Each integer belongs to no more than one class because there can't be two remainders.

\vspace{1cm}

\textit{Fix $n \in \N$ and define $\Z_n$ to be the set of equivalence classes of integers under congruence modulo $n$.  For $[a], [b] \in \Z_n$, we define addition modulo $n$ by $[a] \oplus [b] := [a+b]$, and we define multiplication modulo $n$ by $[a] \odot [b] := [ab]$.}\\

\vspace{1cm}


2.  Prove that addition modulo $n$ is well-defined, i.e., if $a_1 \equiv a_2 \pmod n$ and $b_1 \equiv b_2 \pmod n$, then $a_1 + b_1 \equiv {a_2 + b_2} \pmod n$. (In other words, addition modulo $n$ does not depend on the choice of representative of the equivalence class.)\\

\vspace{0.5cm}

By definition of modular equilavence, $a_1 = a_2 + in$ and $b_1 = b_2 + jn$. So

\begin{align*}
  (a_1 + b_1) - (a_2 + b_2) = in + jn = (i+j)n
\end{align*}

which gives us the desired result after one more application of the definition.

\vspace{1cm}

3.  Prove that multiplication modulo $n$ is well-defined, i.e., if $a_1 \equiv a_2 \pmod n$ and $b_1 \equiv b_2 \pmod n$, then $a_1b_1 \equiv {a_2b_2} \pmod n$. (In other words, multiplication modulo $n$ does not depend on the choice of representative of the equivalence class.)\\

\vspace{0.5cm}

With the same setup as the last problem,

\begin{align*}
  a_1 b_1 -a_2b_2 &= a_2 b_2 + a_2jn + b_2in + injn - a_2b_2\\
  = (a_2j+b_2i+ijn)n
\end{align*}

which is what we need for the definition of modular equivalence.

\textit{Problems 2 and 3 show why it is acceptable to write $\Z_n = \{0, 1, \dots, (n-1)\}$ with operations $+$ and $\cdot$.}\\

\vspace{1cm}

4.  Prove that function composition is associative; that is, if $f: A \rightarrow B$, $g: B \rightarrow C$, and $h: C \rightarrow D$, then $h \circ (g \circ f) = (h \circ g) \circ f$.\\

\vspace{0.5cm}

To prove the composed functions are equal, we prove that for equal input they generate the same output. Indeed:

$(h\circ(g\circ f))(x) = h((g\circ f)(x)) = h(g(f(x))) = (h\circ g)(f(x)) = ((h\circ g)\circ f)(x)$.

\vspace{1cm}

5.  Let $f: A \rightarrow B$ and $g: B \rightarrow C$ be functions.
\begin{itemize}
 \item[(a)] Prove that, if $f$ and $g$ are injective, then the composition of $f$ and $g$ (namely, $g \circ f$) is also injective.

  let $h = g\circ f$. We prove that by the injectivity of $f$ and $g$, $h$ is also injective. Namely, two distinct elements $x_1$ and $x_2$ cannot be mapped to the same $y$. 

  Suppose that $h$ is not injective, such that for some $x_1\neq x_2$, $h(x_1) = h(x_2)$. That means that $g(f(x_1)) = g(f(x_2))$. Because $f$ is injective, this implies that $g(x_1) = g(x_2)$. Because $g$ is injective, we now have $x_1 = x_2$, a contradiction. So $h$ must be injective.
 
 \item[(b)] Prove that, if $f$ and $g$ are surjective, then the composition of $f$ and $g$ (namely, $g \circ f$) is also surjective.

  Let $h = g \circ f$. We prove that surjectivity of $f$ and $g$ implies that for all $y$ in the range of $h$, there exists an $x$ such that $h(x) = y$. Choose any $y$ in the range of $h$. Because $g$ is surjective, there exists a $z$ such that $g(z) = y$. Because $f$ is surjective, there exists an $x$ such that $f(x) = z$.

  So $h(x) = g(f(x)) = y$.
\end{itemize}
\vspace{1cm}

Let $\id_A:A \rightarrow A$ be the \textit{identity function} on $A$, that is, the function defined by $\id_A(a) = a$ for all $a \in A$.  We define a \textit{left inverse} of a function $f: A \rightarrow B$ to be a function $g: B \rightarrow A$ such that $g \circ f = \id_A$.  We define a \textit{right inverse} of a function $f: A \rightarrow B$ to be a function $g: B \rightarrow A$ such that $f \circ g = \id_B$.  An \textit{inverse} (or a \textit{two-sided inverse}) of $f$ is a function that is both a left and a right inverse.\\

\vspace{0.5cm}

6.  Let $f: A \rightarrow B$ be a function.
\begin{itemize}
 \item[(a)] Prove that $f$ is injective if and only if it has a left inverse.

  We first prove that having a left inverse implies injectiveness.
  
  Suppose that $f$ has a left inverse $g$. Since $g\circ f$ is the identity function, for all $x_1, x_2$, $f(x_1) = f(x_2)$ implies $g(f(x_1)) = g(f(x_2))$. But then because $g\circ f$ is the identity function, $x_1 = g(f(x_1)) = g(f(x_2)) = x_2$. Hence having a left-inverse implies injectivity.

  We then prove that injective functions all have left inverses. 

  "$f$ injective" implies that for any $y$ in the \textit{codomain} (as distinct from \textit{range}) of $f$, there is a unique $x$ such that $y=f(x)$. From this fact we construct $g$, the inverse of $f$. For all $y$ in the codomain of $f$, let $g(y) = x$ where $x$ is the unique element of $A$ such that $f(x) = y$.

 \item[(b)] Prove that $f$ is surjective if and only if it has a right inverse.

  Suppose $f$ has a right inverse $g$. We want to prove from this that the image of $f$ covers every element of $B$. I.e. for any $b\in B$, there exists an $a\in A$ such that $f(a) = b$. This is accomplished by chosing $a=g(b)$. Since $g$ is defined all over $B$, we can always find such an $a$. Hence, having a right inverse implies surjectivity.

  Now, suppose $f$ is surjective. We construct a right inverse for $f$. By surjectivity, for any $b\in B$ there exists at least one $a\in A$ such that $f(a) = b$. Define $g(b)$ by choosing any of these $a$ for an input $b$. $g$ covers its entire domain because $f$ covers its entire range, and because for all $b$ we chose $g(b) = a$ such that $f(a) = b$, we guarantee that $f(g(b) = f(a) = b$.

 \item[(c)] Prove that $f$ is bijective if and only if it has an inverse.

  If $f$ has an inverse, by (a) and (b) it would be bijective.

  If $f$ is bijective, by (a) and (b) it would have a left inverse  $g$ and a right inverse $h$. Calling in associativity, we see that $g$ and $h$ has to be the same:
  $ g = g \circ id  = g\circ f \circ h = id\circ h = h$.
  
  And hence $g=h$ is the double-sided inverse of $f$.
\end{itemize}


\end{flushleft}

\end{document}
